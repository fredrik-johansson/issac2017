\documentclass{sig-alternate-05-2015}
\usepackage{paralist}
\usepackage{hyperref}

\begin{document}
\setcopyright{acmcopyright}
\conferenceinfo{ISSAC 2017}{July 25--28, 2017, Kaiserslautern, Germany}

\newtheorem{alg}{Algorithm}
\newtheorem{definition}{Definition}
\newtheorem{Assertion}{Assertion}

\makeatletter
\def\Ddots{\mathinner{\mkern1mu\raise\p@
\vbox{\kern7\p@\hbox{.}}\mkern2mu
\raise4\p@\hbox{.}\mkern2mu\raise7\p@\hbox{.}\mkern1mu}}
\makeatother

%\title{Hecke/Nemo: Number Theory packages for the Julia programming language}
\title{Hecke/Nemo: computer algebra and number theory packages for the Julia programming language}

\numberofauthors{4}
\author{
\alignauthor Claus Fieker\\
   \affaddr{TU Kaiserslautern}\\
   \affaddr{Fachbereich Mathematik, Postfach 3049,}\\
   \affaddr{67653 Kaiserslautern, Germany}\\
   \email{fieker@mathematik.uni-kl.de}
\alignauthor William Hart\\
   \affaddr{TU Kaiserslautern}\\
   \affaddr{Fachbereich Mathematik, Postfach 3049,}\\
   \affaddr{67653 Kaiserslautern, Germany}\\
   \email{goodwillhart@gmail.com}
\alignauthor Tommy Hofmann\\
   \affaddr{TU Kaiserslautern}\\
   \affaddr{Fachbereich Mathematik, Postfach 3049,}\\
   \affaddr{67653 Kaiserslautern, Germany}\\
   \email{thofmann@mathematik.uni-kl.de}
\and
\alignauthor Fredrik Johansson\\
   \affaddr{Inria Bordeaux \&}\\
   \affaddr{Institut de Math\'{e}matiques de Bordeaux} \\
   \affaddr{33400 Talence, France}\\
   \email{fredrik.johansson@gmail.com}
}

\maketitle

\begin{abstract}
We introduce two new packages, Hecke and Nemo, written in the Julia programming language
for computer algebra and number theory.
We demonstrate that high performance generic
algorithms can be implemented in Julia, without the need to resort to a low-level C
implementation. We also describe the various Julia wrappers of existing C/C++ libraries
such as Flint, Arb, Antic and Singular. We give examples of how to use Hecke and Nemo and discuss
some of the algorithms that we have implemented to provide high performance basic
arithmetic.
\end{abstract}


\section{Introduction}




\section{Example}

\section{The Julia programming language}

\section{Nemo: basic arithmetic in Julia}

\section{Hecke: algebraic number theory in Julia}

\section{Generic algorithms in Nemo}

\section{Specific algorithms in Nemo}

\subsection{Flint: number theory}

We envisioned Flint as a set of implementations for specific rings,
but now we realize that it makes sense to use it only for specific
rings where there is some trick

For example, there seems to be no need to have a Flint type for
matrices over $\left(\mathbb{Z}/n\mathbb{Z}\right)[x]$.

Julia has informed development - for example: don't want to just commit suicide
on bad input

Ground rings $\mathbb{Z}$, $\mathbb{Q}$, $\mathbb{Z}/n\mathbb{Z}$,
$\mathbb{F}_q, \mathbb{Q}_p$

dense univariate polynomials, dense matrices
and power series

some flint types require context objects; parent objects hold context objects

algorithms prototyped in Julia, then rewritten in C

showed the way for multivariate polynomials 

\subsection{Arb: arbitrary precision ball arithmetic}

Viable approaches to represent elements of $\mathbb{R}$ and $\mathbb{C}$
include floating-point approximations,
intervals, and lazy bit streams (e.g.\ using a finite-precision
approximation together with an exact symbolic DAG
to allow re-computing the approximation to higher precision).

Nemo includes wrapper code for Arb, which implements real numbers as
arbitrary-precision midpoint-radius intervals (balls) $[m \pm r]$
and complex numbers as rectangular boxes $[a \pm r_1]$ + $[b \pm r_2] i$.

While alternative implementations of $\mathbb{R}$ and $\mathbb{C}$
may be added to Nemo in the future, we envision Arb as a reasonable default,
and we have used it with good success (see Hecke).

For many computer algebra algorithms, the error analysis
necessary to guarantee correct results with floating-point arithmetic
becomes impractical to do by hand.
Interval arithmetic solves this problem by effectively making
error analysis automatic.

With arbitrary-precision interval arithmetic, a form of
coarse-grained lazy evaluation is possible: the user can
try a computation with some tentative precision $p$ and restart
with precision $2p$ if that fails. The precision can be set
optimally when a good estimate for the minimal
required~$p$ is available; that is, the intervals
can be used as if they were plain floating-point numbers, and the automatic
error bounds simply provide a certificate.

A lazy representation for single numbers would be more convenient in some
applications,
but also probably slower (and less space-efficient) due to storing DAGs.
Implementing lazy numbers on top of Arb in Julia
would be an interesting future project.

The midpoint-radius representation used by Arb is particularly efficient
at high precision,
since low precision is sufficient for the radii, which means that
speed and memory efficiency is close to the optimum possible
with plain arbitrary-precision floating-point arithmetic.
Like Flint, Arb systematically uses asymptotically fast algorithms
for operations such as polynomial multiplication, with tuning
for different problem sizes.


ArbField, AcbField

polynomials, matrices

also transcendental functions

\subsection{Antic: algebraic number fields}

\subsection{Singular: commutative algebra}

\section{Future directions}

\section{Acknowledgement}


\begin{thebibliography}{99}

\bibitem{magma}
J. J. Cannon, W. Bosma (Eds.) {\em Handbook of Magma Functions}, Edition 2.13 (2006)

\bibitem{mca} 
J. von zur Gathen and J. Gerhard. {\em Modern Computer Algebra}. Cambridge University Press, 1999.

\bibitem{flint}
W. Hart, F. Johansson and S. Pancratz. {\em FLINT}, open-source C-library. \url{http://www.flintlib.org}

\bibitem{arb}
F. Johansson. {\em Arb}, open-source C-library. \url{http://arblib.org}

\bibitem{ntl}
V. Shoup {\em NTL}, open-source C++ library. \url{http://www.shoup.net/ntl/}

\bibitem{sage}
W. Stein {\em SAGE Mathematics Software}.  \url{http://www.sagemath.org}

\bibitem{julia} J. Bezanson, A. Edelman, S. Karpinski and V. B. Shah. {\em Julia: A fresh approach to numerical computing}. \url{https://arxiv.org/abs/1411.1607}

\bibitem{singular} W. Decker, G. M. Greuel, G. Pfister and H. Sch\"onemann. {\em Singular} --- A computer algebra system for polynomial computations. \url{http://www.singular.uni-kl.de}

\end{thebibliography}
\end{document}
